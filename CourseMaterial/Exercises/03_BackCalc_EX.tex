\documentclass{article}\usepackage[]{graphicx}\usepackage[]{color}
%% maxwidth is the original width if it is less than linewidth
%% otherwise use linewidth (to make sure the graphics do not exceed the margin)
\makeatletter
\def\maxwidth{ %
  \ifdim\Gin@nat@width>\linewidth
    \linewidth
  \else
    \Gin@nat@width
  \fi
}
\makeatother

\definecolor{fgcolor}{rgb}{0.345, 0.345, 0.345}
\newcommand{\hlnum}[1]{\textcolor[rgb]{0.686,0.059,0.569}{#1}}%
\newcommand{\hlstr}[1]{\textcolor[rgb]{0.192,0.494,0.8}{#1}}%
\newcommand{\hlcom}[1]{\textcolor[rgb]{0.678,0.584,0.686}{\textit{#1}}}%
\newcommand{\hlopt}[1]{\textcolor[rgb]{0,0,0}{#1}}%
\newcommand{\hlstd}[1]{\textcolor[rgb]{0.345,0.345,0.345}{#1}}%
\newcommand{\hlkwa}[1]{\textcolor[rgb]{0.161,0.373,0.58}{\textbf{#1}}}%
\newcommand{\hlkwb}[1]{\textcolor[rgb]{0.69,0.353,0.396}{#1}}%
\newcommand{\hlkwc}[1]{\textcolor[rgb]{0.333,0.667,0.333}{#1}}%
\newcommand{\hlkwd}[1]{\textcolor[rgb]{0.737,0.353,0.396}{\textbf{#1}}}%

\usepackage{framed}
\makeatletter
\newenvironment{kframe}{%
 \def\at@end@of@kframe{}%
 \ifinner\ifhmode%
  \def\at@end@of@kframe{\end{minipage}}%
  \begin{minipage}{\columnwidth}%
 \fi\fi%
 \def\FrameCommand##1{\hskip\@totalleftmargin \hskip-\fboxsep
 \colorbox{shadecolor}{##1}\hskip-\fboxsep
     % There is no \\@totalrightmargin, so:
     \hskip-\linewidth \hskip-\@totalleftmargin \hskip\columnwidth}%
 \MakeFramed {\advance\hsize-\width
   \@totalleftmargin\z@ \linewidth\hsize
   \@setminipage}}%
 {\par\unskip\endMakeFramed%
 \at@end@of@kframe}
\makeatother

\definecolor{shadecolor}{rgb}{.97, .97, .97}
\definecolor{messagecolor}{rgb}{0, 0, 0}
\definecolor{warningcolor}{rgb}{1, 0, 1}
\definecolor{errorcolor}{rgb}{1, 0, 0}
\newenvironment{knitrout}{}{} % an empty environment to be redefined in TeX

\usepackage{alltt}
\input{c:/aaaWork/zGnrlLatex/GnrlPreamble}
\input{c:/aaaWork/zGnrlLatex/justRPreamble}
\hypersetup{pdftitle = Vermont R - Back-Calculation}
\newif\ifmakekey
%\renewcommand{\theenumi}{\alph{enumi}}  % changes questions to letters
\IfFileExists{upquote.sty}{\usepackage{upquote}}{}

\begin{document}


%\makekeytrue         % uncomment to show answer key

\section*{Exercise -- Back-Calculation}
Answer the following questions with R code by creating (\textit{and editing if you make a mistake}) an R script and iteratively running the code in RStudio.

\begin{enumerate}
  \item Load the data in the \dfile{MN98WaeYep.csv} file into a data frame in R.


  \item Isolate the Lake Shetek Walleye data.


  \item Is ``plus-growth'' recorded for your chosen data?  Explain.


\ifmakekey
``Plus-growth'' is recorded because one more ``anu'' measurement appears in the data file then the assessed age.  For example, fish 155 was 1-year-old but two radial measurements were recorded and fish 1215 was 6-years-old but seven radial measurements were recorded.
\fi
  \item Reshape the data frame from ``wide'' to ``long'' format so that it will be suitable for adding a back-calculated total length variable.  Make sure to remove unneccesary ``NA''s and the ``plus-growth'', if it was recorded.


  \item Add a variable that is the Fraser-Lee back-calculated total length if the ``correction factor'' is 55 mm.


  \item Compute the mean length-at-back-calculated-age.


\end{enumerate}

\end{document}
