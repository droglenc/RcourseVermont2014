\documentclass{article}\usepackage[]{graphicx}\usepackage[]{color}
%% maxwidth is the original width if it is less than linewidth
%% otherwise use linewidth (to make sure the graphics do not exceed the margin)
\makeatletter
\def\maxwidth{ %
  \ifdim\Gin@nat@width>\linewidth
    \linewidth
  \else
    \Gin@nat@width
  \fi
}
\makeatother

\definecolor{fgcolor}{rgb}{0.345, 0.345, 0.345}
\newcommand{\hlnum}[1]{\textcolor[rgb]{0.686,0.059,0.569}{#1}}%
\newcommand{\hlstr}[1]{\textcolor[rgb]{0.192,0.494,0.8}{#1}}%
\newcommand{\hlcom}[1]{\textcolor[rgb]{0.678,0.584,0.686}{\textit{#1}}}%
\newcommand{\hlopt}[1]{\textcolor[rgb]{0,0,0}{#1}}%
\newcommand{\hlstd}[1]{\textcolor[rgb]{0.345,0.345,0.345}{#1}}%
\newcommand{\hlkwa}[1]{\textcolor[rgb]{0.161,0.373,0.58}{\textbf{#1}}}%
\newcommand{\hlkwb}[1]{\textcolor[rgb]{0.69,0.353,0.396}{#1}}%
\newcommand{\hlkwc}[1]{\textcolor[rgb]{0.333,0.667,0.333}{#1}}%
\newcommand{\hlkwd}[1]{\textcolor[rgb]{0.737,0.353,0.396}{\textbf{#1}}}%

\usepackage{framed}
\makeatletter
\newenvironment{kframe}{%
 \def\at@end@of@kframe{}%
 \ifinner\ifhmode%
  \def\at@end@of@kframe{\end{minipage}}%
  \begin{minipage}{\columnwidth}%
 \fi\fi%
 \def\FrameCommand##1{\hskip\@totalleftmargin \hskip-\fboxsep
 \colorbox{shadecolor}{##1}\hskip-\fboxsep
     % There is no \\@totalrightmargin, so:
     \hskip-\linewidth \hskip-\@totalleftmargin \hskip\columnwidth}%
 \MakeFramed {\advance\hsize-\width
   \@totalleftmargin\z@ \linewidth\hsize
   \@setminipage}}%
 {\par\unskip\endMakeFramed%
 \at@end@of@kframe}
\makeatother

\definecolor{shadecolor}{rgb}{.97, .97, .97}
\definecolor{messagecolor}{rgb}{0, 0, 0}
\definecolor{warningcolor}{rgb}{1, 0, 1}
\definecolor{errorcolor}{rgb}{1, 0, 0}
\newenvironment{knitrout}{}{} % an empty environment to be redefined in TeX

\usepackage{alltt}
\input{c:/aaaWork/zGnrlLatex/GnrlPreamble}
\input{c:/aaaWork/zGnrlLatex/justRPreamble}
\hypersetup{pdftitle = Vermont R - Data Frames}
\newif\ifmakekey
%\renewcommand{\theenumi}{\alph{enumi}}  % changes questions to letters
\IfFileExists{upquote.sty}{\usepackage{upquote}}{}

\begin{document}


%\makekeytrue         % uncomment to show answer key

\section*{Exercise -- Data Frames}
Answer the following questions with R code by creating (\textit{and editing if you make a mistake}) an R script and iteratively running the code in RStudio.

\begin{enumerate}
  \item Load the data in the \dfile{RuffeBio.xlsx} file into a data frame in R.


  \item How many variables are in this data frame?  How many individuals/observations?


\ifmakekey
  There are 10 variables and 40 individuals/observations in this data frame.
\fi
  \item Specifically, what is the name of the first variable
\ifmakekey
\\ The name of the first variable is fishID.
\fi
  \item Show all variables for the fifth individual.


  \item Show all variables for the fifth and seventh individuals.


  \item Show the total lengths for all individuals.


  \item Show ONLY the total length for the seventeenth individual.


  \item Show ONLY the total length for the fifth and seventeenth individuals.


  \item For each situation below, create a new data frame (from the original) and record how many fish are in that data frame.
    \begin{enumerate}
      \item Just female ruffe.


      \item Just ruffe greater than 110 mm.


      \item Just ruffe between 80 and 110 mm.


      \item Excluding all fish of an ``unknown'' sex.


    \end{enumerate}
  \item Create new variables in the original data frame for the following situations.
    \begin{enumerate}
      \item Natural log of length and weight.


      \item Length categories that are 10-mm wide.


      \item Fulton's condition factor (The weight of the fish divided by the cubed length of the fish multiplied by 10000).


    \end{enumerate}
  \vspace{12pt}
  \item If you have time ...
    \begin{enumerate}
      \item Show the length frequency table by sex.


      \item Create a length variable that is the total length in inches.


      \item Create a subset of just male ruffe with a total length less than 100 mm.


      \item What is the \var{tl} for all but the 10th individual?


      \item Show all recorded information for the 11th individual.


    \end{enumerate}
\end{enumerate}

\end{document}
