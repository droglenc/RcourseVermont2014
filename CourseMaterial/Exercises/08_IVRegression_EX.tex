\documentclass{article}\usepackage[]{graphicx}\usepackage[]{color}
%% maxwidth is the original width if it is less than linewidth
%% otherwise use linewidth (to make sure the graphics do not exceed the margin)
\makeatletter
\def\maxwidth{ %
  \ifdim\Gin@nat@width>\linewidth
    \linewidth
  \else
    \Gin@nat@width
  \fi
}
\makeatother

\definecolor{fgcolor}{rgb}{0.345, 0.345, 0.345}
\newcommand{\hlnum}[1]{\textcolor[rgb]{0.686,0.059,0.569}{#1}}%
\newcommand{\hlstr}[1]{\textcolor[rgb]{0.192,0.494,0.8}{#1}}%
\newcommand{\hlcom}[1]{\textcolor[rgb]{0.678,0.584,0.686}{\textit{#1}}}%
\newcommand{\hlopt}[1]{\textcolor[rgb]{0,0,0}{#1}}%
\newcommand{\hlstd}[1]{\textcolor[rgb]{0.345,0.345,0.345}{#1}}%
\newcommand{\hlkwa}[1]{\textcolor[rgb]{0.161,0.373,0.58}{\textbf{#1}}}%
\newcommand{\hlkwb}[1]{\textcolor[rgb]{0.69,0.353,0.396}{#1}}%
\newcommand{\hlkwc}[1]{\textcolor[rgb]{0.333,0.667,0.333}{#1}}%
\newcommand{\hlkwd}[1]{\textcolor[rgb]{0.737,0.353,0.396}{\textbf{#1}}}%

\usepackage{framed}
\makeatletter
\newenvironment{kframe}{%
 \def\at@end@of@kframe{}%
 \ifinner\ifhmode%
  \def\at@end@of@kframe{\end{minipage}}%
  \begin{minipage}{\columnwidth}%
 \fi\fi%
 \def\FrameCommand##1{\hskip\@totalleftmargin \hskip-\fboxsep
 \colorbox{shadecolor}{##1}\hskip-\fboxsep
     % There is no \\@totalrightmargin, so:
     \hskip-\linewidth \hskip-\@totalleftmargin \hskip\columnwidth}%
 \MakeFramed {\advance\hsize-\width
   \@totalleftmargin\z@ \linewidth\hsize
   \@setminipage}}%
 {\par\unskip\endMakeFramed%
 \at@end@of@kframe}
\makeatother

\definecolor{shadecolor}{rgb}{.97, .97, .97}
\definecolor{messagecolor}{rgb}{0, 0, 0}
\definecolor{warningcolor}{rgb}{1, 0, 1}
\definecolor{errorcolor}{rgb}{1, 0, 0}
\newenvironment{knitrout}{}{} % an empty environment to be redefined in TeX

\usepackage{alltt}
\input{c:/aaaWork/zGnrlLatex/GnrlPreamble}
\input{c:/aaaWork/zGnrlLatex/justRPreamble}
\hypersetup{pdftitle = Vermont R - Indicator Variable Regression}
\newif\ifmakekey
%\renewcommand{\theenumi}{\alph{enumi}}  % changes questions to letters
\IfFileExists{upquote.sty}{\usepackage{upquote}}{}

\begin{document}


%\makekeytrue         % uncomment to show answer key

\section*{Exercise -- Indicator Variable Regression}
Answer the following questions with R code by creating (\textit{and editing if you make a mistake}) an R script and iteratively running the code in RStudio.

Consider the following total catches (in 1000s) of Atlantic Cod (\textit{Gadus morhua}) from Gulf of Maine by age group (2-11+) and capture year (1993-2004).  Supposed that the fish are consistently recruited to the gear by age-4 and that consistent catches exist until age-8.

\begin{center}
\begin{tabular}{crrrrrrrrrrrr}
\hline\hline
\widen{-2}{7}{} & \multicolumn{12}{c}{Capture Year} \\ 
\cline{2-13}
\widen{-2}{7}{Age} & 1993 & 1994 & 1995 & 1996 & 1997 & 1998 & 1999 & 2000 & 2001 & 2002 & 2003 & 2004 \\ 
\hline
\widen{0}{5}{2} & 127.8 & 54.0 & 277.0 & 90.0 & 85.4 & 107.5 & 22.1 & 201.1 & 147.2 & 3.0 & 16.4 & 0.9 \\ 
\widen{0}{0}{3} & 2031.8 & 1488.2 & 1169.9 & 630.7 & 495.2 & 482.4 & 647.2 & 534.0 & 1183.5 & 259.5 & 118.6 & 357.8 \\ 
\widen{0}{0}{4} & 783.0 & 1216.6 & 1192.0 & 1936.7 & 455.5 & 597.8 & 568.0 & 828.3 & 685.5 & 884.3 & 442.9 & 249.9 \\ 
\widen{0}{0}{5} & 139.4 & 330.9 & 232.5 & 384.3 & 852.4 & 158.7 & 272.6 & 190.3 & 378.0 & 346.0 & 766.1 & 409.6 \\ 
\widen{0}{0}{6} & 473.8 & 71.0 & 28.6 & 36.9 & 71.4 & 191.4 & 58.0 & 98.9 & 109.1 & 203.5 & 231.4 & 266.0 \\ 
\widen{0}{0}{7} & 29.2 & 85.7 & 13.9 & 4.5 & 5.0 & 26.2 & 49.2 & 16.1 & 59.8 & 81.0 & 103.3 & 74.6 \\ 
\widen{0}{0}{8} & 6.0 & 29.5 & 18.4 & 0.5 & 2.6 & 3.9 & 7.9 & 7.1 & 8.9 & 35.5 & 39.9 & 36.9 \\ 
\widen{0}{0}{9} & 2.0 & 6.7 & 0.8 & 1.3 & 0.3 & 0.4 & 0.0 & 0.0 & 13.3 & 9.5 & 21.7 & 19.3 \\ 
\widen{0}{0}{10} & 0.0 & 0.6 & 1.6 & 0.0 & 0.7 & 1.1 & 4.4 & 0.0 & 1.5 & 9.4 & 9.9 & 11.3 \\ 
\widen{-2}{0}{11+} & 0.0 & 1.2 & 0.2 & 0.0 & 0.1 & 0.4 & 0.0 & 0.0 & 0.5 & 0.6 & 7.4 & 3.5 \\ 
\hline\hline
\end{tabular}
\end{center}

\begin{enumerate}
  \item Identify the earliest and latest year-classes fully represented in these data over the ages consistently fully-recruited and captured by the gear.
\ifmakekey
The earliest and latest year-classes that are fully-represented over ages 4-8 are the 1989 and 1996 year-classes, respectively.
\fi
  \item Enter the catch and age data for the two year-classes from the previous question and the two most intermediate year-classes into Excel in such a manner that you will be able to test if the instantaneous mortality rate differs between any pair of these year-classes.  Save the data and load it into a data frame in R.



  \item Statistically compare the instantaneous mortality rates between the earliest and latest year-classes.  Which year-class, if either, has a higher mortality rate?  By how much?


\ifmakekey
There are statistically different slopes ($p=0.0044$) which implies statistically different mortality rates.  The instantaneous mortality rate for the 1996 year-class is between 0.368 and 1.271 LESS (i.e., shallower slope) than the 1989 year-class.
\fi
  \item Load the \dfile{LakeTroutALTER.csv} file and determine if the length-weight regression is statistically different between male and female fish.


\ifmakekey
Neither the slopes ($p=0.4997$) nor the intercepts ($p=0.0658$) were statistically significantly different between male and female Lake Trout.  Thus, the length-weight relationship for the sexes can be modeled by a single commone line.
\fi
  \item \textit{If time permits ...} Statistically compare the instantaneous mortality rates between the two intermediate year-classes for the Atlantic Cod data.  Which year-class, if either, has a higher mortality rate?  By how much?


\ifmakekey
There are statistically different slopes ($p=0.0104$) which implies statistically different mortality rates.  The instantaneous mortality rate for the 1994 year-class is between 0.100 and 0.499 LESS (i.e., shallower slope) than the 1993 year-class.
\fi
\end{enumerate}

\end{document}
